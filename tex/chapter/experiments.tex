\chapter{Experimental Study}
\section{Experimental Setup}
\subsection{Software and Hardware}
The model was implied using a constraint programming language: MiniZinc 2.5.3. The solver associated is Gecode 6.3.0. Experiments were done on a Macbook Pro 2018 with processor at 2,2 GHz, 6-Core on a Intel Core i7 of 9th generation and a memory of 16 GB, 2400 Mhz DDR4.
\subsection{Instances}
The following table represents instances with their features referred to the number of customers, the number of vehicles, the total capacity available to satisfy customers demands and the total customers demands. Total capacity and total demands are distributed unevenly between customers and vehicles respectively.
\begin{table}[!h]
\label{T:instances}
\begin{center}
\begin{tabular}{| c | c | c | c | c | }
\hline
\textbf{Instance} & \multicolumn{4}{ c |}{\textbf{Features}}  \\ 
\cline{2-5}
& \textbf{\#Customers} & \textbf{\#Vehicles} & \textbf{Tot. Capacity} & \textbf{Tot. Demand}  \\
\hline
% Table values
PR01 & 47 &  4  &  800 & 647 \\ \hline
PR02 & 95 & 20  & 3,900 & 1,210\\ \hline
PR03 & 143 &  20 & 3,800 & 1,765\\ \hline
PR04 & 191 & 20  & 3,700 & 2,472\\ \hline
PR05 & 239 & 20 & 3,600 & 3,335\\ \hline
PR06 & 287 & 20 & 3,700 & 3,665\\ \hline
PR07 & 71 & 20  & 4,000 & 928\\ \hline
PR08 & 143 & 20 & 3,800 & 1,985\\ \hline
PR09 & 215 & 20 & 3,600 & 2,735\\ \hline
PR10 & 287 & 20 & 3,900 & 3,825\\ \hline
PR11 & 47 & 20 & 4,000 & 647\\ \hline \hline

\end{tabular}
\end{center}
\end{table}
\newpage
\subsection{Search Strategy}
The following list represent the search strategy associated with their acronyms carried out on the model in Minizinc with Gecode solver:
\begin{enumerate}
    \item \textbf{SEQDR}: identify the sequential search structured in the following ordered sequence:
    \begin{enumerate}
        \item Take a decision for each element of the vehicles list \begin{math} ve \end{math} using domWdeg and random value heuristics
        \item Take a decision for each element of successors list \begin{math} s \end{math} using domWdeg and random value heuristics
    \end{enumerate} with the Luby Strategy on L = 250
    \item \textbf{SEQSM}: identify the sequential search structured in the following ordered sequence:
    \begin{enumerate}
        \item Take a decision for each element of the vehicles list \begin{math} ve \end{math} using smallest domain and minimum value heuristics
        \item Take a decision for each element of the successors list \begin{math} s \end{math} using domWdeg and random value heuristics
        \end{enumerate} with the Luby Strategy on L = 300.
    \item \textbf{DRLSN}: identify the search structured with domWdeg and random domain heuristic on \begin{math} s \end{math} variables list using Luby Strategy with L = 350 and the Large Neighbourhood Strategy fixing 85\% of the \begin{math}s\end{math} variables list.
    \item \textbf{SMLNS}: identify the SEQ heuristics using Large Neighbourhood Strategy fixing 85\% of the \begin{math} s \end{math} variables list.
    
    
\end{enumerate}
\newpage
\section{Experimental Results}
\subsection{Tables of Results}
\begin{itemize}
    \item Table of Failures
    
        \begin{table}[!h]
    \label{T:instances}
        \begin{center}
        \begin{tabular}{| c | c | c | c | c | c | c | c | c | }
\hline

\textbf{Instance} & \textbf{SEQDR} & \textbf{SEQSM} & \textbf{DRLNS} & \textbf{SMLNS}  \\
\hline
% Failures table
PR01 & 1,617,823  & 1,846,879 &  0 & 0  \\ \hline
PR02 & 261,731  & 276,031  &  0 & 0  \\ \hline
PR03 & 107,615  & 111,227 &  0 & 0 \\ \hline
PR04 & 54,289  & 59,572 &  0 & 0  \\ \hline
PR05 & 25,276  & 35,984 &  0 & 0  \\ \hline
PR06 & 14,075  & 24,078 & 0 & 0 \\ \hline
PR07 & 365,984  & 431,475 &  0 & 0  \\ \hline
PR08 & 101,392  & 104,439 &  0 & 0  \\ \hline
PR09 & 32,350  & 31,886 &  0 & 0  \\ \hline
PR10 & 13,015  & 15,580  &  0 & 0  \\ \hline
PR11 & 738,268  & 491,894 &  0 & 0  \\ \hline
\hline
\end{tabular}
\end{center}
\end{table}

    \item Table of Optimal Solutions
        \begin{table}[!h]
    \label{T:instances}
        \begin{center}
        \begin{tabular}{| c | c | c | c | c | c | c | c | c | }
\hline

\textbf{Instance} & \multicolumn{2}{ c |}{\textbf{SEQDR}} & \multicolumn{2}{ c |}{\textbf{SEQSM}} & \multicolumn{2}{ c |}{\textbf{DRLNS}} & \multicolumn{2}{ c |}{\textbf{SMLNS}}   \\
\cline{2-3}  \cline{4-5} \cline{6-7} \cline{8-9}
& \textbf{Dist.} & \textbf{Vehicl.} & \textbf{Dist.} &\textbf{Vehicl.} & \textbf{Dist.} & \textbf{Vehicl.} & \textbf{Dist.} & \textbf{Vehicl.}  \\
\hline
% Table values
PR01 & 2,395,540  & \textbf{4} &  2,300,229 & \textbf{4} & 0 & \textbf{0} & 0  & \textbf{0}  \\ \hline
PR02 & 5,759,370  & \textbf{20} &  5,171,574 & \textbf{7} & 0 & \textbf{0} & 0  & \textbf{0}  \\ \hline
PR03 & 1,1090,848  & \textbf{20} &  10,433,132 & \textbf{10} & 0 & \textbf{0} & 0  & \textbf{0}  \\ \hline
PR04 & 12,834,190  & \textbf{20} &  12,666,579 & \textbf{14} & 0 & \textbf{0} & 0  & \textbf{0}  \\ \hline
PR05 & 13,939,739  & \textbf{20} & 13,457,287 & \textbf{19} & 0 & \textbf{0} & 0  & \textbf{0}  \\ \hline
PR06 & 18,746,215  & \textbf{20} &  19,188,412 & \textbf{20} & 0 & \textbf{0} & 0  & \textbf{0}  \\ \hline
PR07 & 5,040,291  & \textbf{20} &  4,238,722 & \textbf{5} & 0 & \textbf{0} & 0  & \textbf{0}  \\ \hline
PR08 & 10,159,608  & \textbf{20} &  9,736,174 & \textbf{11} & 0 & \textbf{0} & 0  & \textbf{0}  \\ \hline
PR09 & 15,267,652  & \textbf{20} &  15,056,711 & \textbf{16} & 0 & \textbf{0} & 0  & \textbf{0}  \\ \hline
PR10 & 19,773,277  & \textbf{20} & 19,778,614  & \textbf{20} & 0 & \textbf{0} & 0  & \textbf{0}  \\ \hline
PR11 & 3,171,559  & \textbf{18} &  2,413,564 & \textbf{4} & 0 & \textbf{0} & 0  & \textbf{0}  \\ \hline
\hline
\end{tabular}
\end{center}
\end{table}
\newline
In the table have reported the number of the total distance traveled on the combined route \begin{math} Dist. \equiv td\end{math} and the number of vehicles used \begin{math}Vehicl. \equiv vu\end{math}. The objective value \begin{math} min(f(td,vu)) \end{math} have been calculated using the objective function, \begin{math} tu \end{math} and \begin{math} vu \end{math} defined in the paragraph 2.4. and \begin{math} w = 1000 \end{math}.
\end{itemize}
\newpage
\subsection{Graphical Representation}